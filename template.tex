%%%%%%%%%%%%%%%%%%%%%%%%%%%%%%%%%%%%%%%%%
%
% (c) Lucerne University of Applied Sciences and Arts
%
% HSLU-I: Official Thesis Template
%
% This template complies with the official thesis guidelines released
% on complesis by the department HSLU-I (computer science).
% It supports the languages english and german. The language and thesis style
% can be set by the parameters passed to the documentclass on line 33.
%
% NOTE: The class and style sheets (hsluthesis.cls, hsluthesisterms.sty) should
%       NOT be changed. These are configuration files and define the document style.
%
% Original guidelines can be found on complesis:
% https://complesis.hslu.ch/
%
% Original author:
%   Ramón Christen HSLU-I
%
% Versions:
%   v1.0    21-03-2025: Initial Version
%   v2.0    22-09-2025: Reviewed template structure (.cls added); Extension for continuous education
%   v2.1    29-09-2025: Extended for WIPRO
%   v2.2    10-10-2025: Minor corrections for WIPRO
%
%%%%%%%%%%%%%%%%%%%%%%%%%%%%%%%%%%%%%%%%%


%----------------------------------------------------------------------------------------
%	PACKAGES AND DOCUMENT CONFIGURATIONS
%----------------------------------------------------------------------------------------
% class options (set in \documentclass[...]; order of options is irrelevant):
%   degree:       [wipro, bachelor, master, cas, sas]
%   language:     [english, german]
%   groupwork:    [groupwork] (extends declaration for two students)
%   ip:           [noip] (only for cas/sas if no intellectual property consent is required)
%
% Examples for configuring the documentclass:
% \documentclass[german,master]{hsluthesis}
% \documentclass[english,bachelor]{hsluthesis}
% \documentclass[german,wipro,groupwork]{hsluthesis}
% \documentclass[english,cas,noip]{hsluthesis}
\documentclass[english,bachelor]{hsluthesis}


\usepackage{comment}                            % having comment sections \begin{comment} \end{comment}
\usepackage{amsmath}							% math package
\usepackage{amsfonts}							% font package for math symbols
\usepackage{amssymb}							% symbols package - definition of math symbols
\usepackage{listings}							% package for code representation
\usepackage{graphicx}							% for inclusion of image
\usepackage{subfig}								% to arrange figures next to each other
\usepackage{float}								% text style surrounding images
\usepackage[acronym]{glossaries}         		% package for glossary
\usepackage{tikz}								% used to place logos on title page
% \usepackage{gensymb}							% for special characters such as °
\usepackage[a-1a]{pdfx}                         % Forces PDF/A-1a compliance for long-term archiving
\usepackage{fancyhdr}							% for custom headers and footers

\usepackage{multirow}
\usepackage{siunitx}
\usepackage{tabularx}
\usepackage{tikzscale}

% CSLReferences environment for bibliography
\newlength{\cslhangindent}
\setlength{\cslhangindent}{1.5em}
\newlength{\csllabelwidth}
\setlength{\csllabelwidth}{3em}
\newenvironment{CSLReferences}[2] % #1 hanging-ident, #2 entry spacing
 {% don't indent paragraphs
  \setlength{\parindent}{0pt}
  % turn on hanging indent if param 1 is 1
  \ifodd #1
    \let\oldpar\par
    \def\par{\hangindent=\cslhangindent\oldpar}
  \fi
  % set entry spacing
  \setlength{\parskip}{#2\baselineskip}
  % Redefine \bibitem to work without list environment
  \renewcommand{\bibitem}[2][]{\par}
 }%
 {}
\newcommand{\CSLBlock}[1]{#1\hfill\break}
\newcommand{\CSLLeftMargin}[1]{\parbox[t]{\csllabelwidth}{#1}}
\newcommand{\CSLRightInline}[1]{\parbox[t]{\linewidth - \csllabelwidth}{#1}\break}
\newcommand{\CSLIndent}[1]{\hspace{\cslhangindent}#1}

\hypersetup{hidelinks}                          % hide red border in hyperlinks
\setcounter{tocdepth}{1}                        % hide subsections from TOC
\makenoidxglossaries
\input{acronyms}                                % include acronyms.txt file
\input{glossary}                                % include glossary.txt file
\graphicspath{{figs/}}						    % set path of graphics folder

% Define page styles for proper page numbering
\pagestyle{fancy}
\fancyhf{}                                      % clear all header and footer fields
\fancyfoot[C]{\thepage}                         % centered page number at bottom
\renewcommand{\headrulewidth}{0pt}              % remove header rule
\renewcommand{\footrulewidth}{0pt}              % remove footer rule

% Also apply to plain style (used for chapter pages)
\fancypagestyle{plain}{
  \fancyhf{}
  \fancyfoot[C]{\thepage}
  \renewcommand{\headrulewidth}{0pt}
  \renewcommand{\footrulewidth}{0pt}
}


%%----------------------------------------------------------------------------------------
%%	PDF/A DOCUMENT COMPLIANCE
%%----------------------------------------------------------------------------------------
%\pdfcatalog{
%  /StructTreeRoot <<                            % Define the structure tree root for document tagging
%    /Type /StructTreeRoot                       % Specify that this is a structure tree root
%    /K []                                       % Placeholder for structure elements (empty for now)
%  >>
%  /MarkInfo << /Marked true >>                  % Ensure the document is marked as tagged for accessibility
%}


\begin{document}
%----------------------------------------------------------------------------------------
%	DOCUMENT INFORMATION
%----------------------------------------------------------------------------------------
%\thesisLanguage{english}                        % set thesis language [english, german]
\author{$author$}                            % author name
\city{$city$}                    % author's place of origin
\title{$title$}                            % thesis title
\subtitle{\large $subtitle$}                      % thesis subtitle

\date{$year$}                                     % the year when the thesis was written (used in titlepage)
\defensedate{$defensedate$}                % the date of the private defense
\defencelocation{$defenselocation$}                       % location of defence
\extexpert{$extexpert$}                         % name of external expert
\indpartner{$indpartner$}                       % name of industry partner
\studyprogram{$studyprogram$}                    % name of study program: Business Information Technology, International IT Management, ...

% jury, supervisor and dean are only relevant if acceptance sheet is enabled with the next line
% \addAcceptsheet
\jury{                                          % members of the jury
    \begin{itemize}
        \item Prof. Dr. Name Surname from Lucerne University of Applied Sciences and Arts, Switzerland (President of the Jury);
        \item Prof. Dr. Name Surname from Lucerne University of Applied Sciences and Arts, Switzerland (Thesis Supervisor);
        \item Prof. Dr. Name Surname from Lucerne University of Applied Sciences and Arts, Switzerland (External Expert).
    \end{itemize}
}

\supervisor{$supervisor$}             % name of supervisor
\dean{$dean$}                   % name of faculty dean

\acknowledgments{Thanks to my family, relatives and firends for all the support given to finish this thesis.}



%----------------------------------------------------------------------------------------
%	BEGIN DOCUMENT AND CREATE TITLEPAGE
%----------------------------------------------------------------------------------------
\maketitle


%----------------------------------------------------------------------------------------
%	PREAMBLE
%----------------------------------------------------------------------------------------
$if(abstract)$
\addtocontents{toc}{\protect\setcounter{tocdepth}{-1}}
$abstract$
\addtocontents{toc}{\protect\setcounter{tocdepth}{1}}
$endif$

\tableofcontents

%----------------------------------------------------------------------------------------
%	MAIN CONTENT
%----------------------------------------------------------------------------------------
\mainmatter                                     % Start main content with arabic page numbers
\pagenumbering{arabic}                          % Arabic numerals for main content
\setcounter{page}{1}                            % Reset counter to 1

$body$

$if(appendix)$
$appendix$
$endif$

\listoffigures
\addcontentsline{toc}{chapter}{\listfigurename}

\listoftables
\addcontentsline{toc}{chapter}{\listtablename}

\printnoidxglossary[type=main,title=Glossary]
\addcontentsline{toc}{chapter}{Glossary}

\printnoidxglossary[type=\acronymtype,title=Acronyms]
\addcontentsline{toc}{chapter}{Acronyms}

\glsaddallunused


$if(bibliography)$
$bibliography$
$endif$

\end{document}